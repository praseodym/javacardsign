\documentclass{article}

\usepackage{a4wide}
\usepackage{url}

\begin{document}

\author{Wojciech Mostowski\\
Radboud University Nijmegen\\
Department of Computing Science\\
Nijmegen, the Netherlands\\
e-mail: \url{woj@cs.ru.nl}}

\title{\textbf{The APDU Interface of the PKI Applet}}

\maketitle


\section{Introduction}

This document briefly describes the features and the APDU interface of
the Java Card PKI applet. The applet has been
developed according to the ISO7816 specification, for information not
included here please refer to~\cite{}, or contact the author. For this applet
a certain small selection of options from the ISO standard has been implemented,
as explained below.
The source code of the applet and the host library is available from
the SourceForge SVN repository at:
\begin{center}
\url{https://javacardsign.svn.sourceforge.net/svnroot/javacardsign}
\end{center}
The SourceForge address of the PKI project is
\url{http://javacardsign.sourceforge.net}.  The current code in the project
has been developed mainly by Wojciech Mostowski, \url{woj@cs.ru.nl}.

\section{Applet Specifications}

The current version of the applet and the host library implements the
following features:
\begin{itemize}
\item An ISO7816 file system for storing PKI files according to the Part 15
of the ISO7816 specification: private key directory, certificate directory, CA and user certificates, etc.
It is up to the personalisation software what files will be stored in the applet.
The applet support hierarchical file system including relative to current file selection or
selection by path. Reading of each file can be user PIN protected.
\item PIN and PUC user authentication: a 4--20 characters long PIN code, and a 16 characters
long PUC code. The PUC code lets the user to unblock a forgotten PIN code.
\item The applet does not support any kind of secure messaging for APDU communication.
\item The applet only communicates on the contact interface of the card.
\item Three different cryptographic operations: signing, decryption, and authentication.
Currently, the supported key type and length is RSA 1024 bit.\footnote{This is an ``artificial'' limitation.
The applet could support
keys as long as the underlying Java Card implementation does, but the current personalisation APDU interface limits the 
amount of data when loading up private keys, which in effect limits the key size to 1024 bits. This limitation will
hopefully be lifted in future versions of the applet.} The supported ciphers are the following:
\begin{itemize}
\item for signing (perform security operation command): RSA signature with PKCS1.5 padding and SHA1 or SHA256 digests and RSA signature 
with PSS padding with SHA1 digest.
In all the cases the hashes have to provided ready to the card and in correct format, see APDU interface below,
i.e.\ the card does not do the hashing, as stipulated by the ISO7816-8~\cite{?} specification.\footnote{%
Technical remark: because the hashes are provided ready to the card, the Java Card API \texttt{Signature} API
could not be used. In turn, the PSS padding is done by a manually implemented method in the applet and involves the 
use of \texttt{MessageDigest.ALG\_SHA}.} The result of the signing operation is the RSA signature.
The corresponding Object Identifiers
for the supported algorithms are:
\begin{itemize}
\item OID RSA SHA1 = 1.2.840.113549.1.1.5
\item OID RSA SHA256 = 1.2.840.113549.1.1.11
\item OID RSA PSS = 1.2.840.113549.1.1.10
\item OID SHA1 = 1.3.14.3.2.26
\item OID SHA256 = 2.16.840.1.101.3.4.2.1
\end{itemize}
The Java Card API involved is:
\begin{itemize}
\item \texttt{Cipher.ALG\_RSA\_PKCS1} \texttt{Cipher.ALG\_RSA\_NOPAD} 
\end{itemize}
\item for decryption (perform security operation command): RSA cipher with PKCS1.5 padding. The input for this operation is a valid (i.e.\ properly formatted
and with matching length) encrypted RSA block. The result is the decrypted plain text. The corresponding Object Identifier
for this operation is:
\begin{itemize}
\item OID RSA = 1.2.840.113549.1.1.1
\end{itemize}
The Java Card API involved is:
\begin{itemize}
\item \texttt{Cipher.ALG\_RSA\_PKCS1}
\end{itemize}
\item for authentication (internal authenticate command): RSA cipher with PKCS1.5 padding. The input for this operation
is an arbitrary plain text with in the limits of the supported key length. The output is the encrypted RSA block.
The corresponding Object Identifier
for this operation is:
\begin{itemize}
\item OID RSA = 1.2.840.113549.1.1.1
\end{itemize}
The Java Card API involved is:
\begin{itemize}
\item \texttt{Cipher.ALG\_RSA\_PKCS1}
\end{itemize}
\end{itemize}
\item The AID of the applet is chosen to be \texttt{A000000063504B43532D3135}. The applet does not
  support\slash provide any FCI information on applet\slash file
  selection. It is expected that all such information required for the 
  proper functioning of the host side application is stored in the 
  file system in the form of ISO7816-15 structures (EF.DIR, EF.CIAInfo, EF.OD, EF.CD, etc.).
  The host library and application
  provides an example of initialisation of such structures and 
  uploading them to the card. It is also suggested that the PKI is
  made default selectable on the card.
\end{itemize}

\section{APDU Interface}

Below the APDUs that the applet supports are briefly described.

\subsection{Initialisation}

After a fresh applet is loaded onto the card, the suggested initialisation
sequence is the following:
\begin{enumerate}
\item Select the applet.
\item Set the state of the applet to \textsf{initial}. Applet is in personalisation mode.
\item Optionally, change the historical bytes of the card through the applet. For this the 
applet needs to be default selectable.
\item Load up the file structure information. \textbf{Note:} this step does
not create any files, only provides the intended file structure information.
\item Load up the RSA private keys and their identifiers to the card. Three keys are expected
in total: for signing, decryption, and authentication.
\item Create the contents of all the ISO7816-15 structures intended to 
be on the card (including user certficates matching the keys and CA certificate),
create the corresponding files in the applet
and load up the file contents to the card.
\item Upload the PUC to the card. The PUC is unchangable.
\item Set the state of the applet to \textsf{prepersonalised}.
\item At this stage the personalisation should be finished by setting up the user PIN (PUC has to be provided).
This will put the card into the \textsf{personalised} state. From this point on the personalised state of the
applet is not changable.
\end{enumerate}

The APDUs needed for that are described in the following. 
Personalisation APDUs are only available in the \textsf{initial} state.
On error conditions the APDUs may return a variety of abnormal termination
status words. On success the response APDU is always \texttt{9000}
with no data; during the personalisation phase
no status words other than \texttt{9000} should be accepted.

\subsubsection{Select Applet}

\begin{flushleft}
\begin{tabular}{|l|l|l|l|l|l|l|}
\hline
CLA & INS & P1 & P2 & Lc & Data & Le \\
\hline
\texttt{00} & \texttt{A4} & \texttt{04} & \texttt{00} &
\texttt{0B} & AID=\texttt{A000000063504B43532D3135} & Absent \\
\hline
\end{tabular}
\end{flushleft}


\subsubsection{Put Data -- Set Applet State}

\begin{flushleft}
\begin{tabular}{|l|l|l|l|l|l|l|}
\hline
CLA & INS & P1 & P2 & Lc & Data & Le \\
\hline
\texttt{00} & \texttt{DA} & \texttt{68} & State &
\texttt{00} & Absent & Absent \\
\hline
\end{tabular}
\end{flushleft}
The P2/state can be one of: \texttt{01} initial, \texttt{02} prepersonalised. The state is set to
personalised \texttt{03} implicitly by the change reference data command when setting the user PIN.

\subsubsection{Put Data -- Upload File System Structure Information}

\begin{flushleft}
\begin{tabular}{|l|l|l|l|l|l|l|}
\hline
CLA & INS & P1 & P2 & Lc & Data & Le \\
\hline
\texttt{00} & \texttt{DA} & \texttt{69} & \texttt{00} &
Var & Data with file structure information & Absent \\
\hline
\end{tabular}
\end{flushleft}
The data field in this APDU contains the file structure information according
to the following format. It is a list of concatenated single file information byte strings.
For DF file the following bytes should be present:
\begin{flushleft}
\begin{tabular}{|l|l|l|l|l|l|}
\hline
$b_1$ & $b_2$ & $b_3$ & $b_4$ & $b_5$ & $b_6\dots b_n$ \\
\hline
\texttt{80} ($-1$) & $\mathrm{FID}_{\mathrm{MSB}$ & $\mathrm{FID}_{\mathrm{LSB}$ & 
Parent index & \#children & children indexes\\
\hline
\end{tabular}
\end{flushleft}
The indexes (parent and children) are relative to the begining of the whole data field. For EF files the 
format is this:
\begin{flushleft}
\begin{tabular}{|l|l|l|l|l|}
\hline
$b_1$ & $b_2$ & $b_3$ & $b_4$ & $b_5$  \\
\hline
\texttt{00} & $\mathrm{FID}_{\mathrm{MSB}$ & $\mathrm{FID}_{\mathrm{LSB}$ & 
Parent index & SFI & \\
\hline
\end{tabular}
\end{flushleft}
The SFI byte should be \texttt{00} if no SFI is provided for the file.
And example of a valid file system structure information is this (quote from Java code):
\begin{verbatim}
byte[] fileStructure = {
  -1,            // DF
  0x3F, 0x00,    // FID, MF
  -1,            // no parent
  2, 7, 12,      // two children at indexes 7 and 12
  0,             // EF
  0x2F, 0x00,    // FID, EF.DIR
  0, 0x1E,       // parent at index 0, SFI is 1E
  -1,            // DF
  0x50, 0x15,    // FID, DF.CIA
  0,             // parent at index 0
  9, 26, 31, 36, 41, 46, 51, 56, 61, 66,
                 // 9 children
  0,             // EF
  0x50, 0x32,    // FID, EF.CIAInfo
  12, 0x12,      // parent at index 12, SFI is 12
  0, 0x50, 0x31, 12, 0x11, // EF.OD
  0, 0x42, 0x00, 12, 0x00, // EF.AOD
  0, 0x40, 0x00, 12, 0x00, // EF.PrKD
  0, 0x41, 0x00, 12, 0x00, // EF.CD
  0, 0x41, 0x01, 12, 0x00, // EF.CACert
  0, 0x41, 0x02, 12, 0x00, // EF.UserCert1
  0, 0x41, 0x03, 12, 0x00, // EF.UserCert2
  0, 0x41, 0x04, 12, 0x00, // EF.UserCert3
};
\end{verbatim}

\subsubsection{Put Data -- Set Historical Bytes}

\begin{flushleft}
\begin{tabular}{|l|l|l|l|l|l|l|}
\hline
CLA & INS & P1 & P2 & Lc & Data & Le \\
\hline
\texttt{00} & \texttt{DA} & \texttt{67} & \texttt{00} &
Var & Historical bytes & Absent \\
\hline
\end{tabular}
\end{flushleft}

\subsubsection{Put Data -- Setup Private Key Identifier}

\begin{flushleft}
\begin{tabular}{|l|l|l|l|l|l|l|}
\hline
CLA & INS & P1 & P2 & Lc & Data & Le \\
\hline
\texttt{00} & \texttt{DA} & \texttt{61}/\texttt{62}/\texttt{63} & \texttt{00} &
Var & Key identifier bytes & Absent \\
\hline
\end{tabular}
\end{flushleft}
The P1 byte indicates correspondingly: authentication key, signing key, decryption key. The data field
contains a key identifier, maximum length 16 bytes.

\subsubsection{Put Data -- Upload Private Key}

\begin{flushleft}
\begin{tabular}{|l|l|l|l|l|l|l|}
\hline
CLA & INS & P1 & P2 & Lc & Data & Le \\
\hline
\texttt{00} & \texttt{DA} & \texttt{64}/\texttt{65}/\texttt{66} & Key part &
Var & Corr.\ key part raw data & Absent \\
\hline
\end{tabular}
\end{flushleft}
The P1 byte indicates correspondingly: authentication key, signing key, decryption key. 
The P2 byte indicates the part of the raw data of the key in CRT format in the data field, 
in the following way:\footnote{It is my intention to simplify this key loading to use 
just the modulus and private exponent, i.e.\ the non-CRT form of the key. This will not be part of
the current version of the applet though.}
\begin{flushleft}
\begin{tabular}{|l|l|}
\hline
\texttt{81} & public modulus\\
\texttt{82} & public exponent\\
\texttt{83} & prime number $P$\\
\texttt{84} & prime number $Q$\\
\texttt{85} & prime exponent $P$\\
\texttt{86} & prime exponent $Q$\\
\texttt{87} & CRT coefficient\\
\hline
\end{tabular}
\end{flushleft}

\subsubsection{Create File}

\begin{flushleft}
\begin{tabular}{|l|l|l|l|l|l|l|}
\hline
CLA & INS & P1 & P2 & Lc & Data & Le \\
\hline
\texttt{00} & \texttt{E0} & \texttt{00}} & \texttt{00} &
\texttt{05} & File FID, File length, PIN  & Absent \\
\hline
\end{tabular}
\end{flushleft}
The data field has the following format:
\begin{flushleft}
\begin{tabular}{|l|l|l|l|l|}
\hline
$\mathit{FID}_{\mathrm{MSB}}$ & $\mathit{FID}_{\mathrm{LSB}}$ &
$\mathit{Len}_{\mathrm{MSB}}$ & $\mathit{Len}_{\mathrm{LSB}}$ &
PIN byte \\
\hline
\end{tabular}
\end{flushleft}
where PIN byte indicates whether the file should be PIN protected when reading, \texttt{01} means PIN protect,
\texttt{00} means free read.

\subsubsection{Select File}\label{sec:select}

Absolute, by file identifier \textit{FID} (if none provided, selects MF):
\begin{flushleft}
\begin{tabular}{|l|l|l|l|l|l|l|}
\hline
CLA & INS & P1 & P2 & Lc & Data & Le \\
\hline
\texttt{00} & \texttt{A4} & \texttt{00} & \texttt{00} & \texttt{02}/\texttt{00} &
$\mathit{FID}_{\mathrm{MSB}}$ $\mathit{FID}_{\mathrm{LSB}}$ or Absent & \texttt{00} \\
\hline
\end{tabular}
\end{flushleft}

DF (P1 is \texttt{01}) or EF (P1 is \texttt{02}) under the currently selected file, by file identifier \textit{FID}:
\begin{flushleft}
\begin{tabular}{|l|l|l|l|l|l|l|}
\hline
CLA & INS & P1 & P2 & Lc & Data & Le \\
\hline
\texttt{00} & \texttt{A4} & \texttt{01}/\texttt{02} & \texttt{00} & \texttt{02} &
$\mathit{FID}_{\mathrm{MSB}}$ $\mathit{FID}_{\mathrm{LSB}}$ & \texttt{00} \\
\hline
\end{tabular}
\end{flushleft}


Parent of the current DF:
\begin{flushleft}
\begin{tabular}{|l|l|l|l|l|l|l|}
\hline
CLA & INS & P1 & P2 & Lc & Data & Le \\
\hline
\texttt{00} & \texttt{A4} & \texttt{03} & \texttt{00} & \texttt{00} &
Absent & \texttt{00} \\
\hline
\end{tabular}
\end{flushleft}

By path from the current DF (P1 is \texttt{09}) or MF (P1 is \texttt{08}):
\begin{flushleft}
\begin{tabular}{|l|l|l|l|l|l|l|}
\hline
CLA & INS & P1 & P2 & Lc & Data & Le \\
\hline
\texttt{00} & \texttt{A4} & \texttt{08}/\texttt{09} & \texttt{00} & Var &
The file path & \texttt{00} \\
\hline
\end{tabular}
\end{flushleft}

\subsubsection{Write Binary}

\begin{flushleft}
\begin{tabular}{|l|l|l|l|l|l|l|}
\hline
CLA & INS & P1 & P2 & Lc & Data & Le \\
\hline
\texttt{00} & \texttt{D0} &
$\mathit{Off}_{\mathrm{MSB}}$ or SFI & $\mathit{Off}_{\mathrm{LSB}}$} &
Var & Data to be written to the file & Absent \\
\hline
\end{tabular}
\end{flushleft}
The offset should
to be less or equal than \texttt{7FFF} (most significant bit set to
0). If the most significant bit of P1 is set to 1, then the remaining bits
of P1 indicate SFI, in which case P2 indicates the offset.




\subsubsection{Change Reference Data -- Set PUC}

\begin{flushleft}
\begin{tabular}{|l|l|l|l|l|l|l|}
\hline
CLA & INS & P1 & P2 & Lc & Data & Le \\
\hline
\texttt{00} & \texttt{24} & \texttt{01} & \texttt{00} &
\texttt{10} & PUC data & Absent \\
\hline
\end{tabular}
\end{flushleft}
This form of this command is only available when the applet is in the 
\textsf{initial} state. The data field should contain the ASCII bytes 
of the PUC number (i.e.\ from the range \texttt{30}\dots\texttt{39}).

\subsubsection{Put Data -- CVCA Root Certificate [Opt]}

\begin{flushleft}
\begin{tabular}{|l|l|l|l|l|l|l|}
\hline
CLA & INS & P1 & P2 & Lc & Data & Le \\
\hline
\texttt{00} & \texttt{DA} & \texttt{00} & \texttt{64} &
Var & TLV encoded CVCA root certificate body & Absent \\
\hline
\end{tabular}
\end{flushleft}
The data filed in this command is the CVCA root certificate body
encoded in a TLV structure according to ISO18013 (tag \texttt{7F4E}).

\subsubsection{Put Data -- CA ECDH Private Key [Opt]}

\begin{flushleft}
\begin{tabular}{|l|l|l|l|l|l|l|}
\hline
CLA & INS & P1 & P2 & Lc & Data & Le \\
\hline
\texttt{00} & \texttt{DA} & \texttt{00} & \texttt{63} &
Var & TLV encoded ECDH private key components & Absent \\
\hline
\end{tabular}
\end{flushleft}
The data filed in this command is the concatenated sequence of all the
components comprising an ECDH F2M private key.  Each component is
encoded in its own TLV structure with the following tags. The TLV data
fields are raw corresponding raw octet strings:
\begin{flushleft}
\begin{tabular}{|l|l|}
\hline
\texttt{81} & F2M Field Points (P)\\
\texttt{82} & Curve point A \\
\texttt{83} & Curve point B\\
\texttt{84} & The point G\\
\texttt{85} & The order R\\
\texttt{86} & The private part S\\
\texttt{87} & Cofactor K\\
\hline
\end{tabular}
\end{flushleft}
(These fields match the corresponding set methods in the Java Card
APIs ECPrivateKey).

\subsubsection{Put Data -- Chip Authentication Document Number SICID [Opt]}

\begin{flushleft}
\begin{tabular}{|l|l|l|l|l|l|l|}
\hline CLA & INS & P1 & P2 & Lc & Data & Le \\
\hline \texttt{00} & \texttt{DA} & \texttt{00} & \texttt{65} &
Var & SICID bytes &
Absent \\
\hline
\end{tabular}
\end{flushleft}
The data field of this APDU is the byte representation of the document
number string.

\subsubsection{Put Data -- BAP Key Seed}

\begin{flushleft}
\begin{tabular}{|l|l|l|l|l|l|l|}
\hline
CLA & INS & P1 & P2 & Lc & Data & Le \\
\hline
\texttt{00} & \texttt{DA} & \texttt{00} & \texttt{62} &
\texttt{10} & The key seed string bytes & Absent \\
\hline
\end{tabular}
\end{flushleft}
The key seed string for BAP should be 16 (hex \texttt{10}) characters
long.

\subsubsection{Put Data -- Lock Applet}

\begin{flushleft}
\begin{tabular}{|l|l|l|l|l|l|l|}
\hline
CLA & INS & P1 & P2 & Lc & Data & Le \\
\hline
\texttt{00} & \texttt{DA} & \texttt{DE} & \texttt{AD} &
Absent & Absent & Absent \\
\hline
\end{tabular}
\end{flushleft}
This command finalises the initialisation phase of the applet.

\subsection{Communication}

After the initialisation the normal ISO18013 compliant communication
can be started with the applet. After selection of the applet, the BAP
protocol has to be established (get challenge, and mutual
authenticate), and then files can be read with select file and read
binary commands. If the applet has been EAP enabled, then also EAP
related APDU commands are possible, again according to ISO18013. After
BAP or EAP active authentication is possible with the internal
authenticate command.

\begin{thebibliography}{1}
\bibitem{ISO1} ISO/IEC.  \newblock Personal Identification -- ISO
  Compliant Driving License -- Part 1.  \newblock Technical report,
  2009.  \newblock ISO 18013-1.

\bibitem{ISO2} ISO/IEC.  \newblock Personal Identification -- ISO
  Compliant Driving License -- Part 2.  \newblock Technical report,
  2009.  \newblock ISO 18013-2.

\bibitem{ISO3} ISO/IEC.  \newblock Personal Identification -- ISO
  Compliant Driving License -- Part 3.  \newblock Technical report,
  2009.  \newblock ISO 18013-3.

\end{thebibliography}

\end{document}
